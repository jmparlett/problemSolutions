\documentclass[14pt]{extreport}
\usepackage[utf8]{inputenc}
\usepackage[english]{babel}
\usepackage{amsmath}
\usepackage{amssymb}
\usepackage{fancyhdr}
\usepackage{pgfplots}
\usepackage{setspace}
\usepackage{listings}
\pgfplotsset{compat=1.17} 
% \geometry{a4paper} % or letter or a5paper or ... etc
% \geometry{landscape} % rotated page geometry
\usepackage[margin=2cm]{geometry}
\usepackage[most]{tcolorbox}
\newtcolorbox{tb}[1][]{%
  sharp corners,
  enhanced,
  colback=white,
  height=6cm,
  attach title to upper,
  #1
}

%These setting will make the code areas look Pretty
\lstset{
	escapechar=~,
	numbers=left, 
	%numberstyle=\tiny, 
	stepnumber=1, 
	firstnumber=1,
	%numbersep=5pt,
	language=C,
	% stringstyle=\itfamily,
	%basicstyle=\footnotesize, 
	showstringspaces=false,
	frame=single,
  upquote=true
}

% created 2022-March-23 %

\title{usacoNotes}
\author{Jonathan Parlett}

\begin{document}
\maketitle

\section{Prefix Sums}

The idea of prefix sums is useful when querying a range to figure a certain sum.

We can enumerate all interval sums of a range by calculating the sum from the begining to any point in the the array.

Given
\[ A = [a_0, a_1, ..., a_n] \]

The prefix sum array would be

\[ P = [\sum_0^0, \sum_0^1, \sum_0^2, ..., \sum_0^k] \]

Then interval sum is given by from $[s,e]$ can be obtained from $P[e] - P[s]$.

\end{document}
